\chapter{Escape-Sequences}

The following Escape-Sequences are supported by the VT132 Terminal.

The VT132 in VT100 mode supports sequences from the real DEC VT100, but also some sequences from later models like
VT102, VT220, VT510 or VT520.

Also, sequences from VT52 are available in VT52 mode of the VT100 personality (and using the VT52 personality); these are shown at the end.

Furthermore, some sequences that are used by ANSI.SYS (from DOS) are available.

Sequences from other personalities are not shown in this Guide.

\newpage
\section{VT100(+) sequences}

\begin{tabular}{p{9em} | p{0.66\textwidth}}
\hline
\textbf{Sequence}	& \textbf{Function} \\
\hline
\multicolumn{2}{ l }{Cursor Movement:} \\
\hline
ESC [ \textit{row} ; \textit{column} H		& Move cursor to position \\
ESC [ \textit{row} ; \textit{column} f		& Move cursor to position \\
ESC [ \textit{n} D	& Cursor Backwards \\
ESC [ \textit{n} B	& Cursor Down \\
ESC [ \textit{n} C	& Cursor Forward \\
ESC [ \textit{n} A	& Cursor Up \\
ESC [ \textit{n} d	& Move cursor to the indicated row \\
ESC [ \textit{n} G	& Move cursor to the indicated column \\
ESC 7				& Save Cursor position and attributes \\
ESC 8				& Restore Cursor position attributes \\
ESC D				& Index (move cursor down or scroll) \\
ESC E				& Move cursor to start of next line \\
ESC M				& Reverse Index \\
\hline
\multicolumn{2}{ l }{Text insertion / deletion:} \\
\hline
ESC [ J				& Erase screen from cursor down \\
ESC [ 0 J			& Erase screen from cursor down \\
ESC [ 1 J			& Erase screen from cursor up \\
ESC [ 2 J			& Erase entire screen \\
ESC [ K				& Erase line from cursor right \\
ESC [ 0 K			& Erase line from cursor right \\
ESC [ 1 K			& Erase line from cursor left \\
ESC [ 2 K			& Erase entire line \\
ESC [ \textit{n} P	& Delete \textit{n} Characters on Current Line \\
ESC [ \textit{n} X	& Erase \textit{n} Characters on Current Line \\
ESC [ \textit{n} @	& Insert \textit{n} Characters (Spaces) \\
\hline
ESC [ \textit{n} L	& Insert \textit{n} Lines \\
ESC [ \textit{n} M	& Delete \textit{n} Lines \\
\hline
\end{tabular}

\begin{tabular}{p{9em} | p{0.66\textwidth}}
\hline
\textbf{Sequence}		& \textbf{Function} \\
\hline
\multicolumn{2}{ l }{Scrolling:} \\
\hline
ESC [ \textit{n}; \textit{n} r	& Set Top and Bottom line of scroll region \\
ESC [ \textit{n} S				& Scroll Up \textit{n} Lines \\
ESC [ \textit{n} T				& Scroll Down \textit{n} Lines \\
\hline
\multicolumn{2}{ l }{Tab stops:} \\
\hline
ESC H		& Set tab stop at cursor \\
ESC [ g		& Clear tab stop at cursor \\
ESC [ 0 g	& Clear tab stop at cursor \\
ESC [ 3 g	& Clear all tab stops \\
\hline
\multicolumn{2}{ l }{Text Attributes:} \\
\hline
ESC \# 3	& Double Height Line - Top Half \\
ESC \# 4	& Double Height Line -- Bottom Half \\
ESC \# 5	& Single Width / Single Height Line \\
ESC \# 6	& Double Width / Single Height Line \\
\hline
ESC [ 0 m	& Attributes Off \\
ESC [ 1 m	& Bold or Increased Intensity \\
ESC [ 2 m	& Dim (Decreased Intensity) \\
ESC [ 22 m	& Disable Bold and Dim \\
ESC [ 3 m	& Italic \\
ESC [ 23 m	& Disable Italic \\
ESC [ 4 m	& Underscore \\
ESC [ 24 m	& Disable Underline \\
ESC [ 5 m	& Blink \\
ESC [ 25 m	& Disable Blink \\
ESC [ 7 m	& Inverse \\
ESC [ 27 m	& Disable Inverse \\
ESC [ 8 m	& Blank (Invisible) \\
ESC [ 28 m	& Disable Blank \\
\hline
\end{tabular}


\begin{tabular}{p{9em} | p{0.66\textwidth}}
\hline
\textbf{Sequence}		& \textbf{Function} \\
\hline
\multicolumn{2}{ l }{Color:} \\
\hline
ESC [ \textit{n} m		& Set Foreground Color (\textit{n} = 30\dots37) \\
ESC [ 39 m				& Set Current Foreground Color as Default \\
ESC [ \textit{n} m		& Set Background Color (\textit{n} = 40\dots47) \\
ESC [ 49 m				& Set Background Color as Default \\
ESC [ \textit{n} m		& Set Bright Foreground Color (\textit{n} = 90\dots97) \\
ESC [ \textit{n} m		& Set Bright Background Color (\textit{n} = 100\dots107) \\
\hline
\multicolumn{2}{ l }{Color numbers:} \\
\hline
0						& Black \\
1						& Red \\
2						& Green \\
3						& Yellow (Brown) \\
4						& Blue \\
5						& Magenta \\
6						& Cyan \\
7						& White \\
\hline
\multicolumn{2}{ l }{Selective Erase:} \\
\hline
ESC [ 1 " q				& Protect from Selective Erase \\
ESC [ 0 " q				& Unprotect from Selective Erase \\
ESC [ 2 " q				& Unprotect Selective Erase \\
ESC [ ? \textit{s} J	& Selective Erase on Screen (like ESC[J) \\
						& \textit{s}: 0: to end of screen, 1: to top of screen, \\
						& 2: whole screen \\
ESC [ ? \textit{s} K	& Selective Erase in Line (like ESC[K above) \\
						& \textit{s}: 0: to left, 1: to right, 2: whole line \\
\hline
\end{tabular}

\begin{tabular}{p{9em} | p{0.66\textwidth}}
\hline
\textbf{Sequence}	& \textbf{Function} \\
\hline
\multicolumn{2}{ l }{Switches, \textit{x} = h / l to enable / disable:} \\
\hline
ESC [ 4 \textit{x}		& Insert Mode \\
ESC [ 20 \textit{x}		& $\dagger$ Line Feed/New Line Mode \\
ESC [ ? 1 \textit{x}	& Cursor Keys Mode \\
ESC [ ? 2 l				& $\dagger$ Start VT52 mode (ESC < returns) \\
ESC [ ? 3 \textit{x}	& $\ddagger$ 132/80 Column Mode \\
ESC [ ? 4 \textit{x}	& $\dagger$ Scrolling Mode \\
ESC [ ? 5 \textit{x}	& $\dagger$ Screen Mode \\
ESC [ ? 6 \textit{x}	& Origin Mode \\
ESC [ ? 7 \textit{x}	& $\dagger$ Autowrap Mode \\
ESC [ ? 8 \textit{x}	& $\dagger$ Auto Repeat Mode \\
ESC [ ? 9 \textit{x}	& $\dagger$ Interlace Mode \\
ESC [ ? 12 \textit{x}	& Set Blinking Cursor \\
ESC [ ? 25 \textit{x}	& Set Cursor Visible \\
ESC [ ? 40 \textit{x}	& Allow switching 80/132 Mode \\
ESC [ ? 42 \textit{x}	& NRCS Mode \\
ESC [ ? 45 \textit{x}	& Reverse-wraparound Mode \\
ESC [ ? 47 \textit{x}	& Use Alternate Screen Buffer \\
ESC [ ? 67 \textit{x}	& Backarrow Key Mode \\
ESC [ ? 1047 \textit{x}	& Use Alternate Screen Buffer \\
ESC [ ? 1048 \textit{x}	& Save/Restore Cursor \\
ESC [ ? 1049 \textit{x}	& S/R Cursor and Use Alternate Screen \\
\hline
ESC =	& Keypad Application Mode \\
ESC >	& Keypad Numeric Mode \\
\hline
\end{tabular}

\par\noindent\rule{2in}{0.4pt}\\
\footnotesize{
	$\ddagger$ Controls an option on Set-Up A, see section \vref{setupA} \\
	$\dagger$ Controls an option on Set-Up B, see section \vref{setupB}
}


\begin{tabular}{p{9em} | p{0.66\textwidth}}
\hline
\textbf{Sequence}	& \textbf{Function} \\
\hline
\multicolumn{2}{ l }{Set Cursor Style ( \textvisiblespace{} = Space, 0x32):} \\
\hline
ESC [ 0 \textvisiblespace{} q	& Blinking Block \\
ESC [ 1 \textvisiblespace{} q	& Blinking Block (default) \\
ESC [ 2 \textvisiblespace{} q	& Steady Block \\
ESC [ 3 \textvisiblespace{} q	& Blinking Underline \\
ESC [ 4 \textvisiblespace{} q	& Steady Underline \\
ESC [ 5 \textvisiblespace{} q	& Blinking Bar \\
ESC [ 6 \textvisiblespace{} q	& Steady Bar \\
\hline
ESC c		& Reset (like Power Cycle) \\
ESC [ !		& Soft Terminal Reset \\
ESC \# 8	& Screen Alignment Display (fill screen with \textit{E}s) \\
ESC \textvisiblespace{} F	& Send C1 Control Character as 7-Bit Escaped Characters \\
ESC \textvisiblespace{} G	& Send C1 Control Character as 8-Bit Characters \\
ESC [ \textit{x} " p	& Select Conformance Level\footnotemark \\
\hline
\multicolumn{2}{ l }{Select Character Set:} \\
\hline
ESC ( A				& G0 - Unitied Kingdom Set \\
ESC ( B				& G0 - ASCII Set \\
ESC ( 0				& G0 - Special Grapahics \\
ESC ( 1				& G0 - Alt Standard Character Set \\
ESC ( 2				& G0 - Alt Special Graphics \\
ESC ) A				& G1 - Unitied Kingdom Set \\
ESC ) B				& G1 - ASCII Set \\
ESC ) 0				& G1 - Special Grapahics \\
ESC ) 1				& G1 - Alt Standard Character Set \\
ESC ) 2				& G1 - Alt Special Graphics \\
ESC * \textit{n}	& Select Character Set G2 (\textit{n} = A,B,0,1,2) \\
ESC + \textit{n}	& Select Character Set G3 (\textit{n} = A,B,0,1,2) \\
\hline
\end{tabular}
\footnotetext{For details, see VT510 manual: \scriptsize{\url{https://www.vt100.net/docs/vt510-rm/DECSCL.html}}}

\begin{tabular}{p{9em} | p{0.66\textwidth}}
\hline
\textbf{Sequence}	& \textbf{Function} \\
\hline
\multicolumn{2}{ l }{Terminal Query Sequences:} \\
\hline
ESC [ 6 n	& Request Cursor Position \\
ESC [ \textit{row} ; \textit{column} R & \textit{To Host:} Cursor Position Report \\
\hline
ESC [ 5 n	& Request Operating Status \\
ESC [ 0 n	& \textit{To Host:} Terminal in good operating condition \\
ESC [ 3 n	& \textit{To Host:} Terminal has a malfunction \\
\hline
ESC Z		& Identify Terminal \\
ESC [ c		& Query Device Atttibutes \\
ESC [ 0 c	& Query Device Atttibutes \\
ESC [ ? 1 ; \textit{n} c	& \textit{To Host:} Device Attributes \\
	& VT132 will reply: ESC [ ? 64 ; 1 ; 6 ; 22 c \\
	& \textit{64 = class 4 device, 1 = 132 columns,} \\
	& \textit{6 = Selective Erase, 22 = ?} \\ % TODO
\hline
\end{tabular}

\newpage
\section{VT52 sequences}

These VT52 sequences are available in VT100 mode:

\begin{tabular}{p{9em} | p{0.66\textwidth}}
\hline
\textbf{Sequence}	& \textbf{Function} \\
\hline
ESC A	& Cursor Up \\
ESC B	& Cursor Down \\
ESC C	& Cursor Right \\
ESC D	& Cursor Left \\
ESC F	& Enter Graphics Mode \\
ESC G	& Exit Graphics Mode \\
ESC H	& Cursor to Home \\
ESC I	& Reverse Line Feed \\
ESC J	& Erase to End of Screen \\
ESC K	& Erase to End of Line \\
ESC Y \textit{row column}	& Move cursor to position \\
ESC Z	& Identify \\
ESC / Z	& \textit{To host:} response \\
ESC =	& Enter Alternate Keypad Mode \\
ESC >	& Exit Alternate Keypad Mode \\
ESC <	& Enter ANSI Mode \\
\hline
\end{tabular}

\section{ANSI.SYS sequences}

\begin{tabular}{p{9em} | p{0.66\textwidth}}
\hline
\textbf{Sequence}	& \textbf{Function} \\
\hline
ESC [ s	& save cursor state (position and attributes) \\
ESC [ u	& restore cursor state (position and attributes) \\
\hline
\end{tabular}
