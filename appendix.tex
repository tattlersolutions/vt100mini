\appendix

\chapter{Hardware}

\section{Installation}

To install the VT132 in an RC2014 or a compatible system, place it in a standard or extended bus socket.

If your backplane does not offer an extended bus, the modem cannot be used via bus pins, instead the modem FTDI header has to be used.
The extended bus pins are not used for other purposes.

\textbf{On a new VT132, the NVR (Non-volatile RAM) is not initialized.}

At power-on or reset, following the \textbf{Wait} message, a new VT132 will display \textbf{Error} along with a series of bells.
This is expected because the NVR has not been initialized.
Once you write settings to the NVR, this error should not persist.

\section{Jumper settings and headers}
\label{jumperheaders}

The board features the following jumpers:

\begin{tabular}{p{7em} | p{0.71\textwidth}}
\hline
\textbf{Jumper} & \textbf{Function} \\
\hline
JP1 RxA & \multirow{2}{*}{Connect terminal to port A on bus pins} \\
JP2 TxA \\
\hline
JP3 RxB & \multirow{2}{*}{Connect modem to port B on ext. bus pins} \\
JP4 RxB \\
\hline
JP5 Pwr FTDI & Connect Power to FTDI +5V pin \\
\hline
JP6 Pwr Modem & Connect Power to modem FTDI +5V pin \\
\hline
\end{tabular}

Enabling JP1 and JP2 is highly advised, as the VT100 terminal serial connection is not available on a FTDI header.

If you want to provide power to, or take power from either of the 6 pin headers then JP5 and JP6 will provide power
or isolate the \texttt{Vcc} pin in the \texttt{FTDI Program} and \texttt{Modem Port B} 6-pin headers respectively.

\textbf{Warning:} You should normally only connect one power source to the system at a time.

\begin{tabular}{ c | p{0.38\textwidth} || c | p{0.38\textwidth}}
\hline
\multicolumn{2}{ l || }{FTDI Program} & \multicolumn{2}{ l }{Modem Port B} \\
\hline
\textbf{Pin} & \textbf{Function} & \textbf{Pin} & \textbf{Function} \\
\hline
1 & GND             & 1 & GND \\
2 & not connected   & 2 & CTS \\
3 & Vcc (+5V)       & 3 & Vcc (+5V) \\
4 & Tx              & 4 & Tx \\
5 & Rx              & 5 & Rx \\
6 & not connected   & 6 & RTS \\
\hline
\end{tabular}

If your RC2014 serial module uses the RTS/CTS pins, you can disable JP3+4 and use jumper wires to connect the \textbf{Modem Port B} of the VT132 to your serial module,
as no bus pins are assigned to RTS/CTS.

The \textbf{FTDI Program} header outputs debug messages from the ESP32 microcontroller. You can connect another terminal (or a PC) to watch the debug output.
The output uses 115.200 baud, 8 data bits, no parity, 1 stop bit (8-N-1).

It is also possible to flash the ESP32 via this header. To enable the \textit{programming mode}, press and hold \textbf{Reset}, press and hold \textbf{Prog}, release \textbf{Reset}
and release \textbf{Reset}. Another firmware can now be uploaded, eg. via \texttt{esptool}.

\newpage
\section{Buttons}

The VT132 module offers two buttons:

\begin{itemize}
 \item Reset
 \item Prog
\end{itemize}

The hardware \textbf{Reset} button on the PCB reboots the ESP32 (EN line reset). This will cause both the VT100 terminal and the modem to reset.
Any unsaved settings (terminal and/or modem settings) will be lost.

\textit{Note:} The computer connected to the VT132 will not be reset, so after the reset, you will be in the same program as before.

The hardware \textbf{Prog} button is used to switch the baud rate of the modem. After each press of this button, the modem outputs its new
baud rate to the serial port - so you can press this button repeatedly until you can read your baud rate.

