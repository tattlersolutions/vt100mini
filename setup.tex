\chapter{Set-Up Mode}

\section{Keys in Set-Up Mode}

\begin{tabular}{p{6em} | p{0.75\textwidth}}
\hline
\textbf{Key} & \textbf{Function} \\
\hline
\LKeyF{1}		& Show/hide help \\
\LKey{5}		& Advance to the next screen \\
\LKeyShiftX{T}	& Reset tab stops to default \\
\LKeyTab		& Move cursor to the next tab stop \\
\LKeyEnter		& Move cursor to the beginning of the line \\
\LKeyShiftX{C}	& On Set-Up B: reset NVRAM to factory defaults on next boot \\
\hline
\end{tabular}

\section{The Set-Up Screens}

The Set-Up screens imitate the same functions on a real DEC VT100 terminal.

Use \LKeyF{1} to toggle the help display which shows all keys. \\
Use \LKey{5} to advance to the next screen.

\subsection{Set-Up A}

This page shows the tab stops and offers to toggle \textbf{80/132 columns} text display and also switching between
\textbf{online} and \textbf{local mode}.

In \textbf{online mode}, the terminal is connected to the computer via the serial port.\\
In \textbf{local mode}, the terminal can be used to connect directly to the modem using the \textit{Quick Settings menu} (see \vref{quicksettings}).

Use \LKeyShiftX{S} (capital \texttt{S}) to save and \LKeyShiftX{R} (capital \texttt{R}) to recall the settings from NVS.

\subsection{Set-Up B}

On this page, various configuration \textit{bits} can be set. Use the cursor to navigate above the bit and press \LKey{6} to toggle it. Use \LKeyTab{} and \LKeyEnter{} to move quicker.

These bits are available:

\begin{tabular}{p{8em} | p{0.68\textwidth}}
\hline
\textbf{Bit}	& \textbf{Function} \\
\hline
Scroll			& Use smooth scrolling \\
Auto Repeat		& Press longer on a key and the input will be repeated \\
Screen Inv.		& Invert the screen colors (to eg. black on white) \\
Cursor			& Change cursor shape (block / line) \\
\hline
Margin Bell		& Ring the bell when cursor is on right margin \\
Keyclick		& Every keypress will make a clicking sound \\
Ansi/VT52		& \dots \\
Auto Xon/off	& Use Xon/Xoff as flow control \\
\hline
US/UK			& Switch between US and UK keyboard layout \\
Wrap Around		& \dots \\
New Line		& \dots \\
Interlace		& Enable a 'scanline effect' \\
\hline
Parity Odd/Even	& Not used \\
Parity			& Not used \\
Bits 7/8		& Not used \\
Backspace DEL/BS	& Send DEL or BS when pressing \texttt{Backspace} \\
\hline
Bold is Bright	& If bold text is displayed bright \\
Bold is Thick	& If bold text is displayed \textbf{thick} \\
Home on Erase	& Should the cursor go to upper left on clear screen \\
NumLock on Reset	& Should NumLock be enabled on bootup \\
\hline
\end{tabular}

\newpage
\textbf{Please note:}
\begin{itemize}
 \item Use keys \LKey{7} and \LKey{8} to set the baudrate used on the terminal serial port, it is shown in the lower right.
 \item The bits for \textbf{bold} text are applied to text with the attribute \texttt{ESC[1m}.
 \item \textit{Home on Erase} makes the cursor go home (\texttt{ESC [H}) on a clear screen request (\texttt{ESC[2J}), just as MS-DOS \texttt{ANSI.SYS} works.
\end{itemize}

\subsection{Set-Up C}

This screen is a VT132 enhancement over the VT100 functionality. \\
You can also see the version number and the memory utilization (with help disabled).

You can change the following settings:

\begin{tabular}{ c | p{0.88\textwidth}}
\hline
\textbf{Key} & \textbf{Function} \\
\hline
\LKey{2}	& Select codepage (DEC or Codepage 437) \\
\LKey{3}	& Select lines per screen (24, 25 or 30) \\
\LKey{6}	& Toggle ANSI or VGA color palette \\
\LKey{7}	& Set default foreground color \\
\LKey{8}	& Set default background color \\
\hline
\end{tabular}

\textbf{Please note:}
\begin{itemize}
 \item Codepage 437 is not available in 132 column mode.
 \item When switching to 30 line mode, the modem part of VT132 is not available due to memory constraints.
 \item To set the colors, move the cursor above the desired color on the bottom left of the screen.
 \item The current default foreground/background color is displayed by the word \texttt{Default} on the last line of the screen.
 \item The current screen size is displayed on the last line of the screen.
\end{itemize}

\subsection{Set-Up D}

This screen is a VT132 enhancement over the VT100 functionality.

\begin{tabular}{ c | p{0.85\textwidth}}
\hline
\textbf{Key} & \textbf{Function} \\
\hline
\LKey{6}	& Set the keyboard layout \\
\LKey{7}	& Toggle MCS/NRCS \\
\hline
\LKeyUp		& \multirow{2}{*}{Change Personality} \\
\LKeyDown	& \\
\hline
\end{tabular}

\textbf{Please note:}
\begin{itemize}
 \item To change the keyboard layout, move the cursor above the desired label and press \LKey{6}.
 \item When enabling NRCS, certain characters in the lower 127 characters are replaced with country-specific characters according to the
		selected keyboard layout.
 \item The current keyboard layout and personality is shown in the last line of the screen.
 \item Selecting \textit{WordStar/VT100} will enable WordStar cursor movement sequences for the cursor keys instead of standard ANSI
		cursor sequences \textit{(ANSI/VT100)}.
\end{itemize}

