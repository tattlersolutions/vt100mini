\chapter{The Terminal}

The main function of the VT132 is the terminal. It functions as a VT100 terminal but is also capable of
ANSI color sequences and can also display DOS \texttt{ANSI.SYS} compatible sequences.\\
Alternative personalities like ADM-3A or VT-52 are also available.

\section{Keys in Terminal Mode}

\begin{tabular}{p{6em} | p{0.75\textwidth}}
\hline
\textbf{Key}				& \textbf{Function} \\
\hline
\LKeyAlt + \biolinum{SysRq}	& Open Setup Screen \\
\LKeyAlt + \LKeyEsc			& Open Quick Settings Menu (left \LKeyAlt only)\\
\LKeyCtrlX{J}				& Send \texttt{LINEFEED} key \\
\biolinum{ScrollLock}		& \texttt{NO} \texttt{SCROLL} function, like \LKeyCtrlX{S} / \LKeyCtrlX{Q} \\
\LKeyAlt$^\dagger$ + \LKeyPageUp	& Open Scroll History (see below) \\
\hline
\multicolumn{2}{ l }{In Scroll History:} \\
\hline
\LKeyPageUp / \LKeyPageDown	& Scroll back/forward one screen \\
\LKeyUp / \LKeyDown			& Scroll back/forward one line \\
\LKeyEsc					& Quit Scroll History and return to current screen \\
\LKeyShiftX{C}				& Clear Scroll History and quit to current screen \\
\hline
\end{tabular}

When scrolling past the bottom of the Scroll History, the current screen will be shown again.

Any change of screen parameters (80/132 columns, 24/25/30 lines per screen, DEC/CP437 font) will clear the Scroll History.

\vspace*{\fill}
\noindent\rule{2in}{0.4pt}\\
{\footnotesize
$^\dagger$: left \LKeyAlt key
}

\section{The Quick Settings Menu}
\label{quicksettings}

Use these keys to navigate inside this menu:

\begin{tabular}{p{6em} | p{0.75\textwidth}}
\hline
\textbf{Key}	& \textbf{Function} \\
\hline
\LKeyEsc		& Close Quick Settings Menu (at top level)\\
\hline
\LKeySpace		& \multirow{3}{*}{Make selection, enter menu} \\
\LKeyEnter		& \\
\LKeyRight		& \\
\hline
\LKeyEsc		& \multirow{3}{*}{Go back one level} \\
\LKeyBack		& \\
\LKeyLeft		& \\
\hline
\end{tabular}

\begin{itemize}[leftmargin=1em]
 \item Selecting an action (ie. not a menu or checkbox/radio button) will execute this action and close the menu.
 \item You can press the \underline{underlined} key to select an option.
\end{itemize}

The following \textbf{menu options} are available:

\begin{itemize}[leftmargin=1em]
 \item Actions:
 \begin{itemize}[noitemsep]
  \item Clear display
  \item Soft reset
  \item Reset terminal: The same as \texttt{0} in Set-Up menus, resets the terminal to a defined state.
  \item Reboot: Reboots the microcontroller of the VT132
  \item Clean NVR: \textit{factory reset} the VT132 by deleting all terminal and modem settings
 \end{itemize}
 \item Display:
 \begin{itemize}[noitemsep]
  \item Lines of history: Enables the Scroll History of 100, 1000 or 5000 (= default) lines.
  \item CRT saver: Enables a screensaver after 10 seconds or 1, 5, 10, 20 or 30 minutes of inactivity.
  \item Clear history: clear the Scroll History.
 \end{itemize}
\newpage
 \item Terminal Type:
 \begin{itemize}[noitemsep]
  \item Emulation Mode: change personality
  \item 7-bit NRCS characters: enable/disable NRCS\footnote{National Replacement Character Set: When enabled, some characters are replaced with language specific 
	characters, eg. Umlauts in German} mode
 \end{itemize}
 \item ASCII emulation:
 \begin{itemize}[noitemsep]
  \item Select a pre-defined emulation \textit{profile}
  \item White/Green/Amber/Blue screen all sets: DEC Codepage, Bold = Bright + Thick, ANSI color palette (except Amber: VGA palette), 80x24 screen
  \item Ansi.sys sets: Codepage 437, Home on Clear, VGA color palette, 80x25 screen
 \end{itemize}
 \item Keyboard: Select the keyboard layout (US/UK/German/Italian)
 \item Modem: Enable modem locally
 \begin{itemize}[noitemsep]
  \item When enabling this setting, the modem part cannot be accessed via the serial connection
  \item Instead, when switching from ONLINE to LOCAL mode, you can talk directly to the modem (bypassing the connected computer)
 \end{itemize}
 \item On-Line: switch between ONLINE and LOCAL mode (for using the modem, see above)
 \item Save Settings: Write settings to NVRAM (same function as in the Set-Up Screens)
 \item Restore Settings: Load settings from NVRAM
\end{itemize}

